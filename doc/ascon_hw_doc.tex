\documentclass[runningheads]{llncs}

\usepackage{url}
\usepackage{graphicx}
\usepackage{algorithm,algpseudocode}
\usepackage{eqparbox} %
\usepackage{listings}
\usepackage{multicol} %
\usepackage{multirow}
\usepackage[usenames,dvipsnames]{xcolor} %
\usepackage{amsmath,amssymb}
\usepackage{array} %
\usepackage{tikz} %
\usepackage{pgfplots}
\usepackage{relsize} %
%\usepackage{footnote}
%\makesavenoteenv{tabular}
\usepackage{color, colortbl} %
\usepackage{textcomp} 
%\usepackage{times}

% \usepackage{draftwatermark}
% \SetWatermarkText{Draft \today -- do not distribute!}

% to avoid over-full lines
\tolerance=2000

\usepackage{bold-extra}
\usepackage{booktabs}
\usepackage{gensymb}

\usetikzlibrary{patterns}
 
\title{{\scshape Ascon}-128 Hardware Design Document \\ v0.9} %{\rm v\oldstylenums{0.8}}}

\author{Erich Wenger \and Hannes Gro{\ss}}
\institute{Graz University of Technology\\
Institute for Applied Information Processing and Communications\\
Inffeldgasse 16a, 8010 Graz, Austria\\
\email{\{Erich.Wenger,Hannes.Gross\}@iaik.tugraz.at}%
}

\pagestyle{plain}
\let\maketitleorig\maketitle
\renewcommand{\maketitle}{\maketitleorig\thispagestyle{plain}}

%--- Macros
\newcommand{\todo}[1]%
  {\marginpar{\baselineskip0ex\rule{2,5cm}{0.5pt}\\[0ex]{\tiny\textsf{#1}}}}
\newcommand{\eg}{e.g.}
\newcommand{\ie}{\textsl{i.e.}}
\newcommand{\etal}{\textsl{et al.}}

\definecolor{ColorArea}{rgb}{1,1,0.8}
\newcolumntype{A}{>{\columncolor{ColorArea}}r}
\definecolor{ColorRuntime}{rgb}{0.9,0.9,1}
\newcolumntype{R}{>{\columncolor{ColorRuntime}}r}
\definecolor{ColorPower}{rgb}{1,0.9,0.9}
\newcolumntype{P}{>{\columncolor{ColorPower}}r}
\definecolor{ColorEnergy}{rgb}{0.9,1,0.9}
\newcolumntype{E}{>{\columncolor{ColorEnergy}}r}

\algrenewcommand{\algorithmicrequire}{\textbf{Input:}}
\algrenewcommand{\algorithmicensure}{\textbf{Output:}}
\algrenewcommand{\algorithmiccomment}[1]{\hskip0.5em$\triangleright$ #1}
\algnewcommand{\Lineif}[2]{%
    %\STATE\algorithmicif\ {#1}\ \algorithmicthen\ {#2} \algorithmicend\ \algorithmicif%
    \State\algorithmicif\ {#1}\ \algorithmicthen\ {#2}
}


\begin{document}

\maketitle

\section{Introduction}

This document defines the interface and discusses design decisions for our implementations of the authenticated encryption algorithm \textsc{Ascon}-128.  The \textsc{Ascon} family was designed by Dobraunig, Eichlseder, Mendel and Schl{\"a}ffer~\cite{dobraunig2014ascon} and submitted as a candidate to the CAESAR competition for authenticated encryption designs.
Our goal was to explore the design space of hardware implementations of \textsc{Ascon}-128, and we provide both low-area and high-speed implementations.  Within this paper, we document the common hardware bus interface, as well as the memory layout of all our \textsc{Ascon}-128 implementations.  In Section~\ref{sec:results}, we check the implementations for their hardware characteristics. 

\section{Assumptions}

Researchers found many ways to let their hardware implementations shine using many different semi-advisable techniques.
In the following, we discuss the different techniques.

\begin{description}
\item[Technology.] Hardware results highly depend on the used manufacturing technology. As a comparison in terms of $\mu m^2$ of designs manufactured in different technologies is not possible, researchers use the unit Gate Equivalents (GE). A GE is defined as the size of a 2-input NAND gate with 0.5 drive. This normalized unit makes the chip area more comparable than ${\mu}m^2$. However, there are technologies with bigger NAND gates and there are technologies with smaller NAND gates. Let's guess which technology is preferred by researchers...
\item[Interface.] Having a design without bus interface might reduce the size and the complexity of a hardware implementation significantly. However, there exist only a very limited number of applications that use custom architectures that may not be based on a microprocessor or do not use a synchronous bus interface. Therefore, one has to question the significance of a general purpose hardware implementation without bus interface.
\item[``Signal is magically created by test bench'' or ``key comes from outside''.] Testing is one of the most critical practical requirements of most implementations. Usually that is done with a test bench that is written in VHDL, Verilog, or TCL. The test bench is the source of the key, random numbers, the plaintext, etc. However, in the most generic case, both the key and the random numbers have to be written via the bus.
\item[RAM/ROM Macros] are a eligible option for many practical implementations. However, for better comparison at least the synthesis results without using macros should be reported. Also, even a proper use of latches to save area is \emph{not} good practice.
\item[Toolchain.] Cadence, Synopsis, Mentor offer differently clever tools with different advantages and merits. As it can hardly be expected that a researcher has access to all design flows, a minimum is to state the used design flow and the used versions of the tools.
\item[Post synthesis vs. post place-and-route.] It is common practice to report post-synthesis area results. However, a designer has to make sure that her design will be actually routable without significant area overhead (e.g., less than ten percent).
\item[Maximum clock frequencies] are highly dependent on the used technology, the operating temperature, and the used toolchain. For high-throughput designs it is reasonable to assume at least a 100\,MHz clock. Multiple clock domains usually want to be avoided and are only used when absolutely necessary. For resource constrained implementations, a minimum maximum-clock-frequency of 10\,MHz in a ``modern'' technology seems reasonable.
  For reproducibility of the results, the temperature range of the used hardware library should be given.
\item[Comparisons] of different implementations done in different technologies, with different toolchains, etc., must be doubted for their fairness. A fair comparison can only be done if all designs were made under nearly-identical assumptions and tools. Within the CAESAR competition, it is advisable to publish the used VHDL/Verilog code such that other researchers may verify the results and use it for fair comparisons. 
\end{description}

\section{Bus Interface}

A synchronous 32-bit bus interface is used for all our implementations. Although there exist potentially interesting microprocessors with smaller memory interfaces (e.g., the Atmel AVR ATmega series or the Texas Instruments MSP430 series) and some dedicated applications (e.g., Internet-of-Things, RFID, Wireless Sensor Networks) with very unique requirements, the majority of industry will use a 32-bit processor with a synchronous bus interface. Anyways, it is straightforward to strip away the 32-bit bus interface and replace it with the custom interface needed for the custom design goals. 

\subsection{Bus Signals}

Table~\ref{tbl:bus-signals} specifies the interface signals to access the hardware accelerator. It is closely related to the AXI-lite bus specification~\cite{} with some minor differences that simplify the development of the designs competing within the CAESAR competition.  

The bus has a synchronous interface with the following use cases:
\begin{itemize}
\item A global asynchronous active-low reset sets all registers to their initial value. 
\item If \texttt{CSxSI} and \texttt{WExSI} are `1', then \texttt{DataWritexDI} is written to \texttt{AddressxDI} when the clock changes from `0' to `1'.
\item If \texttt{CSxSI} is `1' and \texttt{WExSI} is `0', then \texttt{AddressxDI} is used to update \texttt{DataReadxDO} asynchronously.
  There can only be one asynchronous read operation per cycle.
\end{itemize}

Thus every read and every write operation is finished in a single clock cycle. The AXI-lite bus always uses two consecutive cycles to perform read and write operations. The modification of the specification from above to support the AXI-lite specification is straightforward: add registers to \texttt{CSxSI}, \texttt{WExSI}, and \texttt{AddressxDI} to introduce a one-cycle delay and add an output register to \texttt{DataReadxDO}.

\begin{table}[htb]
\caption{Bus signals used for 32-bit bus interface.}
\label{tbl:bus-signals}
\centering
%\begin{tabular}{|l|c|l|} \hline
\begin{tabular}{l@{\hskip 16pt}c@{\hskip 16pt}l} \toprule
Name & I/O & Functionality \\ \midrule %\hline
\texttt{ClkxCI}  & in & clock signal \\
\texttt{RstxRBI} & in & asynchronous active-low reset \\
\texttt{CSxSI}   & in & active-high chip select/enable \\
\texttt{WExSI}   & in & high: write, low: read \\
\texttt{AddressxDI}   & in & 20-bit word address signal \\
\texttt{DataWritexDI} & in & 32-bit data write bus \\
\texttt{DataReadxDO}  & out & 32-bit data read bus \\
  \bottomrule %\hline
\end{tabular}
\end{table}%

\subsection{Memory Layout}

In general, an Authentication Encryption with Associated Data (AEAD) hardware design must support the following functionality: 

\begin{itemize}
\item Initialization
\item Processing associated data
\item Encryption of plaintext
\item Decryption of ciphertext
\item Finalization and tag generation
\end{itemize}

\newcommand{\specialcell}[2][c]{%
  \begin{tabular}[#1]{@{}l@{}}#2\end{tabular}}

    In particular, the interface for our \textsc{Ascon}-128 hardware designs is shown in Table~\ref{tbl:mem-layout}.

\begin{table}[htb]
\caption{Memory layout.}
\label{tbl:mem-layout}
\centering
%\begin{tabular}{|l|c|l|} \hline
\begin{tabular}{l@{\hskip 16pt}c@{\hskip 16pt}l} \toprule
Word Address & R/W & Functionality \\ \midrule %\hline
\texttt{0000} & read & \texttt{0xdeadbeef} to check basic connectivity \\
\texttt{0001} & read & Status Register: reads 1 if busy, otherwise 0 \\
\texttt{0002} & write & \specialcell[t]{Control Register:\\- bit 0: Initialize\\- bit 1: Associate\\- bit 2: Encrypt\\- bit 3: Decrypt\\- bit 4: Final Encrypt\\- bit 5: Final Decrypt\\- bit 6: Final Associate} \\
\texttt{0003} & write & \specialcell[t]{Scheduled Status Register:\\reads 1 if operation scheduled, otherwise 0 \\(only with \textsc{Ascon}-fast)} \\
\texttt{0004-0007} & write & 128-bit key, little endian \\
\texttt{0008-000B} & write & 128-bit nonce, little endian \\
\texttt{000C-000D} & write & 64-bit data to encrypt/decrypt/associate, little endian \\
\texttt{000C-000D} & read & 64-bit encrypted/decrypted data, little endian \\
\texttt{0010-0013} & read & 128-bit tag, little endian \\
\bottomrule %\hline
\end{tabular}
\end{table}%

\subsection{Software Access}

The usage of the interface within software will work as follows. Listing~\ref{lst:c_struct} defines the structure to access the hardware. Listing~\ref{lst:c_auth_enc} converts the interface from the CAESAR competition to access the proposed hardware designs. \texttt{CAST\_B\_I} and  \texttt{SET\_I\_B} are two marcros that convert the endianness. For an ease of pseudo-code, Listing~\ref{lst:c_auth_enc} does not handle the padding of data and does not perform wait-and-start optimizations (where CPU and HW work in parallel). 

\begin{lstlisting}[float=htb,language=C,caption={Memory Interface defined in C. Accessible at address 0x40000000.},label=lst:c_struct,basicstyle=\footnotesize\ttfamily]
typedef struct ASCON128_INTERFACE_ {
    uint32_t unique_value;
    uint32_t status;
    uint32_t control;
    uint32_t scheduled_status;
    uint32_t key[4];
    uint32_t nonce[4];
    uint32_t data[2];
    uint32_t filler2[2];
    uint32_t tag[4]; 
} ASCON128_INTERFACE;

volatile ASCON128_INTERFACE* interface = (volatile ASCON128_INTERFACE*) 0x40000000;

#define CTRL_INIT (1 << 0)
#define CTRL_ASSOCIATE (1 << 1)
#define CTRL_ENCRYPT (1 << 2)
#define CTRL_DECRYPT (1 << 3)
#define CTRL_ENCRYPT_FINAL (1 << 4)
#define CTRL_DECRYPT_FINAL (1 << 5)
#define CTRL_ASSOCIATE_FINAL (1 << 6)
\end{lstlisting}

\begin{lstlisting}[float=htb,language=c,caption={Simple way of using the hardware to authenticate and encrypt data.},label=lst:c_auth_enc,basicstyle=\footnotesize\ttfamily]
int crypto_aead_encrypt(
  uint8_t *c, uint64_t *clen,
  const uint8_t *m, uint64_t mlen,
  const uint8_t *ad, uint64_t adlen,
  const uint8_t *nsec,
  const uint8_t *npub,
  const uint8_t *k) {
    *clen = mlen + 16;
    interface->key[0] = CAST_B_I(k+0);
    interface->key[1] = CAST_B_I(k+4);
    interface->key[2] = CAST_B_I(k+8);
    interface->key[3] = CAST_B_I(k+12);
    interface->nonce[0] = CAST_B_I(npub+0);
    interface->nonce[1] = CAST_B_I(npub+4);
    interface->nonce[2] = CAST_B_I(npub+8);
    interface->nonce[3] = CAST_B_I(npub+12);
    interface->control = CTRL_INIT;
    while(interface->status != 0);

    // TODO: specially handle padding and adlen % 8 != 0
    while(adlen >= 8) {
        interface->data[0] = CAST_B_I(ad); ad += 4;
        interface->data[1] = CAST_B_I(ad); ad += 4; adlen -= 8;
        interface->control = CTRL_ASSOCIATE;
        while(interface->status != 0);
    }
    interface->control = CTRL_ASSOCIATE_FINAL; // single cycle

    // TODO: specially handle padding and mlen % 8 != 0
    while(mlen > 8) {
        interface->data[0] = CAST_B_I(m); m+=4;
        interface->data[1] = CAST_B_I(m); m+=4;
        interface->control = CTRL_ENCRYPT;
        while(interface->status != 0);
        SET_I_B(c, interface->data[0]); c+=4;
        SET_I_B(c, interface->data[1]); c+=4;
    }
    interface->data[0] = CAST_B_I(m); m+=4;
    interface->data[1] = CAST_B_I(m); m+=4;
    interface->control = CTRL_ENCRYPT_FINAL;
    while(interface->status != 0);
    SET_I_B(c, interface->data[0]); c+=4;
    SET_I_B(c, interface->data[1]); c+=4;
    SET_I_B(c, interface->tag[0]); c+=4;
    SET_I_B(c, interface->tag[1]); c+=4;
    SET_I_B(c, interface->tag[2]); c+=4;
    SET_I_B(c, interface->tag[3]); 
}
\end{lstlisting}

\section{Hardware Designs}

\subsection{\textsc{Ascon}-fast}

\textsc{Ascon}-fast was designed for maximum throughput per given area.
\textsc{Ascon}-fast performs a single round transformation in a single cycle. It comes with an I/O buffer such that it is possible to read/write the I/O buffer while the round transformations are performed. As the data can be read/written while the round transformations are active, the hardware is always fully utilized and no significant timing overhead for the interface is necessary. 

\subsection{\textsc{Ascon}-small-64bit}

\textsc{Ascon}-small-64bit uses a 64-bit data path that is able to either XOR, AND, NOT, or ROTATE one or two operands. Therefore the data-path is quite straightforward and consists of only two multiplexers, the computation unit, and some logic to update the register contents. Naturally, the data-path is kept busy by a state-machine that is interwoven with the external interface. This design also comes with an I/O buffer, otherwise data to decrypt would have to be written twice (write ciphertext, read plaintext, write ciphertext again). 

\subsection{\textsc{Ascon}-xlow-area}

\textsc{Ascon}-xlow-area is an uncompromisingly trimmed implementation that strictly avoids redundancies of functional units in the data path, e.g., only a single 5-bit S-box instance is implemented. The main part of the design---the so-called state---is therefore organized as five 64-bit linear feedback shift registers (LFSR) that are shifted through the functional units. The results are then either directly fed back into the state registers or temporarily stored into another LFSR. In order to keep the control path as simple as possible, the whole state encoding is implicitly done by a 13-bit counter. The counter is thus subdivided into three parts counting the number of processed rounds and the number of processed bits in each iteration. Furthermore, the counter selects the function that is applied on the state in the current iteration. The rest of the control path is formed by only a few gates which are required to handle functionality that is not applied in each round.


\section{Results} \label{sec:results}

The hardware was synthesized using Cadence Encounter v08.10 and a UMC 90\,nm manufacturing technology. The main runtime and area results are summarized in Table~\ref{tbl:hw-char}.
The given maximum frequencies were generated using  the worst-case operating
conditions: 0.9\,V power supply and a temperature of 125\celsius.

\begin{table}[htb]
  \caption{Characteristics of \textsc{Ascon}-128 hardware implementations.}
\label{tbl:hw-char}
\centering
%\begin{tabular}{|l|r|r|r|} \hline
\begin{tabular}{l@{\hskip 12pt}r@{\hskip 12pt}r@{\hskip 12pt}r} \toprule
  Design & \textsc{Ascon}-fast & \textsc{Ascon}-small-64bit & \textsc{Ascon}-xlow-area \\
  \midrule
  Area                             & 8,895\,GE            & 6,110\,GE               & 3,994\,GE                 \\
  Cycles per round                 & 1                    & 59                      & 512                       \\
  Cycles per block plus interface  & $1 \times 6 + 0 = 6$ & $59 \times 6 + 6 = 360$ & $512 \times 6 + 6 = 3078$ \\
  Cycles per byte                  & 0.75                 & 45                      & 384.75                    \\
  Maximum frequency                & 300\,MHz             & 230\,MHz                & 380\,MHz                  \\
  Maximum throughput               & 400\,MByte/sec       & 5.1\,MByte/sec          & 988\,kByte/sec            \\
  \bottomrule %\hline
\end{tabular}
\end{table}%


% \section*{Acknowledgments}

\vspace*{-.5cm} % TODO remove asap -- silly hack

%--- Bibliography
\newpage
\bibliographystyle{abbrv}
\bibliography{bibliography} 

\appendix
%\newpage


\end{document}
